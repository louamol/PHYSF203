\documentclass[11pt,a4paper,oneside]{article}

\usepackage{import}

\import{Packages/}{custom_packages.tex}
\import{Packages/}{custom_macros.tex}

\title{Particule dans un puits de potentiel}
\author{Louan Mol}

\begin{document}

\begin{center}
    {\huge \textbf{Particule dans un puits de potentiel}}
\end{center}

\vspace{1cm}

Le cas d'une particule dans un puits de potentiel est en mécanique quantique est très important et peut-être perturbant lorsque l'on essaye d'aller dans les détails. En particulier, c'est une situation simple sur laquelle on peut s'entrainer à bien comprendre l'utilisation des syémtries.

\paragraph{Énoncé.} Considérons une particule de masse $m$ à une dimension dans un \emph{puits de potentiel} de largeur $L$:
\begin{equation}
    V(x)=
    \begin{cases}
        0,\quad &\text{ si } x\in]-L/2,L/2[,\\
        +\infty,\quad &\text{ si } x\notin]-L/2,L/2[.
    \end{cases}
\end{equation}
On dit souvent que la particule est dans une boîte (unidimensionnelle). Nous cherchons à calculer les états stationnaires de ce système. L'équation à résoudre est donc
\begin{equation}
    -\frac{\hbar^2}{2m}\dv[2]{}{x}\vp(x)=E\vp(x),\qquad E\in\R \label{eq:vp}
\end{equation}
où $\vp$ est la partie spatiale de la fonction d'onde, c'est-à-dire que la fonction d'onde est $\psi(x,t)=\vp(x)e^{-\frac{i}{\hbar}Et}$. La solution de \eqref{eq:vp} doit satisfaire deux propriétés:
\begin{enumerate}
    \item \textbf{Conditions aux bords}: le fait que le potentiel soit infini hors du puits implique que la finction d'onde doit s'annuler dans cette partie de l'espace et donc que
    \begin{align}
        \psi(L/2) &= 0,\\
        \psi(-L/2) &= 0.
    \end{align}
    \item \textbf{Normalisation} : par définition, une fonction d'onde doit être normée, donc
    \begin{equation}
        \int_{-L/2}^{L/2}\abs{\vp(x)}^2\d x = 1.
    \end{equation}
\end{enumerate}


\paragraph*{Symétrie.} En analysant le potentiel, on peut voir que ce dernier est symétrique sur la transformation de coordonnées
\begin{equation}
    x\mapsto-x.
\end{equation}
Comme le terme cinétique est également invariant sous cette transformation, l'Hamiltonien au complet est invariant sous cette transformation. Cette dernière est donc une symétrie du système.
... 

En général, les syémtries d'un problème constituent un outil très important pour mieux comprendre ce dernier. Dans ntore cas les syémtries sont très pratiques pour simplifier la résolution du problème. Cepedant, il doit aussi être possible en principe, de résoudre le problème sans utiliser les symétries. Par exemple, si l'on n'a pas remarqué qu'une syémtrie était présente.

Dans la suite, nous allons résoudre le problème de trois manière différentes : d'abord sans utiliser les syémtries, et ensuite en utilisant les symétries, avec deux approches différentes.

\section{Résolution sans symétrie}

\section{Résolution avec symétrie : méthode A}

\section{Résolution avec symétrie : méthode B}

\section{Ajout d'une barrière infiniment maince}





\end{document}