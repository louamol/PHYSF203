\documentclass[11pt,a4paper,oneside]{article}

\usepackage{import}

\import{Packages/}{custom_packages.tex}
\import{Packages/}{custom_macros.tex}

\title{Particule dans un puits de potentiel}
\author{Louan Mol}

\begin{document}

\begin{center}
    {\huge \textbf{Particule dans un puits de potentiel}}
\end{center}

\vspace{1cm}

Le cas d'une particule dans un puits de potentiel en mécanique quantique est très important. Dans ces notes, nous allons énoncer plusieurs méthodes pour résoudre ce problème et clarifier les différences entre ces dernières.

\paragraph{Énoncé.} Considérons une particule de masse $m$ à une dimension dans un \emph{puits de potentiel} de largeur $L$:
\begin{equation}
    V(x)=
    \begin{cases}
        0,\quad &\text{ si } x\in]-L/2,L/2[,\\
        +\infty,\quad &\text{ si } x\notin]-L/2,L/2[.
    \end{cases}
\end{equation}
On dit souvent que la particule est dans une boîte (unidimensionnelle). Nous cherchons à calculer les états stationnaires de ce système. Si $\psi(x,t)$ est un fonction d'onde correspondant à un état stationnaire d'énergie $E$, alors $\psi(x,t)$ est nécéssairement de la forme 
\begin{equation}
    \psi(x,t)=\vp(x)e^{-\frac{i}{\hbar}Et}.
\end{equation}
et l'équation de Schrödinger impose à la partie spatiale $\vp(x)$ de satisfaire l'équation
\begin{equation}
    -\frac{\hbar^2}{2m}\dv[2]{}{x}\vp(x)=E\vp(x),\qquad E\in\R,
\end{equation}
Cette dernière peut être réécrite comme
\begin{equation}
    \dv[2]{}{x}\vp(x)=-k^2\vp(x),\qquad E\in\R \label{eq:vp}
\end{equation}

La solution de \eqref{eq:vp} doit satisfaire deux propriétés:
\begin{enumerate}
    \item \textbf{Conditions aux bords}: le fait que le potentiel soit infini hors du puits implique que la fonction d'onde doit s'annuler dans cette partie de l'espace et donc que
    \begin{align}
        \psi(L/2) &= 0,\\
        \psi(-L/2) &= 0.
    \end{align}
    \item \textbf{Normalisation} : par définition, une fonction d'onde doit être normée, donc
    \begin{equation}
        \int_{-L/2}^{L/2}\abs{\vp(x)}^2\d x = 1.
    \end{equation}
\end{enumerate}


\paragraph*{Symétrie.} En analysant le potentiel, on peut voir que ce dernier est inchangé sous la transformation de coordonnées
\begin{equation}
    x\mapsto-x.
\end{equation}
Comme le terme cinétique est également invariant sous cette transformation, l'Hamiltonien au complet est invariant sous cette transformation. Cette dernière est donc une symétrie du système.

Les symétries d'un système sont une propriété très importante de ce dernier. Avoir connaissance de ces dernières permet souvent de simplifier la résolution d'un problème. Dans ce cas-ci on peut résoudre le problème tout aussi facilement sans utiliser ses symétries.

Dans la suite, nous allons résoudre le problème de trois manière différentes : d'abord sans utiliser les symétries, et ensuite en utilisant les symétries, avec deux approches différentes.

\section{Résolution sans symétrie}

\begin{enumerate}
    \item Commeçons par résoudre l'équation \eqref{eq:vp} dans la région $[-L/2,L/2]$. La solution générale est
    \begin{equation}
        \vp(x)=Ae^{ikx}+Be^{-ikx},\qquad A,B\in\C.
    \end{equation}
    \item Imposons les conditions aux bords:
    \begin{align}
        \vp(L/2) &= 0 \Rightarrow B = -Ae^{ikL},\\
        \vp(-L/2) &= 0 \Rightarrow k=\frac{\pi}{L}n\equiv k_n,\quad n\in \N.
    \end{align}
    À ce stade, la solution est de la forme
    \begin{align}
        \vp_n(x) &= A(e^{ik_nx}+(-1)^{n+1}e^{-ik_nx})\\
        &= 
        \begin{cases}
            2A\cos(k_nx),\quad \text{si $n$ est impair},\\
            2Ai\sin(k_nx),\quad \text{si $n$ est pair}.
        \end{cases}
    \end{align}
    \item Pour finir, il ne reste plus qu'à normer la solution. La constante de normalization dépend de la parité de $n$. On trouve
    \begin{equation}
        A=
        \begin{cases}
            \frac{1}{\sqrt{2L}} ,\quad \text{si $n$ est impair},\\
            -\frac{i}{\sqrt{2L}} ,\quad \text{si $n$ est pair}.
        \end{cases}
    \end{equation}
\end{enumerate}
Pour finir, les solutions stationnaires sont données par l'ensemble des fonctions d'ondes
\begin{equation}
    \vp_n(x) = 
        \begin{cases}
            \sqrt{\frac{2}{L}}\cos(k_nx),\quad \text{si $n$ est impair}\\
            \sqrt{\frac{2}{L}}\sin(k_nx),\quad \text{si $n$ est pair}
        \end{cases},\qquad n\in\N.\label{eq:solfin}
\end{equation}

\section{Résolution avec symétrie}

\subsection{Méthode A}

\begin{enumerate}
    \item On résout l'équation \eqref{eq:vp} dans la région $[-L/2,L/2]$. La solution générale est
    \begin{equation}
        \vp(x)=Ae^{ikx}+Be^{-ikx},\qquad A,B\in\C.
    \end{equation}
    \item On impose que la solution soit paire ou impaire:
    \begin{align}
        \text{Paire} &: \vp(x)=\vp(-x) \Rightarrow A=B,\text{ donc } \vp^\P(x) = 2A^\P\cos(kx)\\
        \text{Impaire} &: \vp(x)=-\vp(-x) \Rightarrow A=-B,\text{ donc } \vp^\I(x) = 2iA^\I\sin(kx)
    \end{align}
    \item On impose les conditions aux bords:
    \begin{align}
        \vp^\P(L/2)&=0 \Rightarrow k=\frac{\pi}{L}(2n+1)\equiv k^\P_n (n\in\N)\\
        \vp^\I(L/2)&=0 \Rightarrow k=\frac{\pi}{L}2n\equiv k^\I_n (n\in\N_0)
    \end{align}
    Il n'est nécessaire d'imposer qu'une des deux conditions car, les fonctions étant paires ou impaires, l'autre condition sera automatiquement satisfaite.
    \item Finalement, nous imposons que les fonction d'onde soient normées, on trouve
    \begin{align}
        A^\P &= \frac{1}{\sqrt{2L}},\\
        A^\I &= -\frac{i}{\sqrt{2L}}.
    \end{align}
\end{enumerate}
Nous avons trouvé les solutions suivantes:
\begin{align}
    \vp_n^\P(x) &= \sqrt{\frac{2}{L}}\cos(k^\P_nx),\qquad (n\in\N)\label{eq:solf1}\\
    \vp_n^\I(x) &= \sqrt{\frac{2}{L}}\sin(k^\I_nx),\qquad (n\in\N_0).\label{eq:solf2}
\end{align}

\subsection{Méthode B}

\begin{enumerate}
    \item On résout l'équation \eqref{eq:vp} dans la région $[0,L/2]$. La solution générale est
    \begin{equation}
        \vp(x)=Ae^{ikx}+Be^{-ikx},\qquad A,B\in\C.
    \end{equation}
    \item On prolonge cette solution sur tout $[-L/2,L/2]$. Il existe deux manières de la prolonger:
    \begin{align}
        \text{Paire} &: \vp^\P(x)=
        \begin{cases}
            A^\P e^{-ikx}+B^\P e^{ikx}, x<0\\
            A^\P e^{ikx}+B^\P e^{-ikx}, x>0
        \end{cases}\\
        \text{Impaire} &: \vp^\I(x)=
        \begin{cases}
            -A^\I e^{-ikx}-B^\I e^{ikx}, x<0\\
            A^\I e^{ikx}+B^\I e^{-ikx}, x>0.
        \end{cases}
    \end{align}
    \item Après l'étape précédente, il faut s'assurer que la fonction résultante est bien $C^1$ partout, en particulier en $x=0$. Commençons par imposer que les solutions soient continues:
    \begin{align}
        \vp^\P(0^+)-\vp^\P(0^-)&=0 : \text{ toujours satisfait}\\
        \vp^\P(0^+)-\vp^\I(0^-)&=0 \Rightarrow A^\I = -B^\I
    \end{align}
    et ensuite que leur dérivée soient continues:
    \begin{align}
        \p_x\vp^\P(0^+)-\p_x\vp^\P(0^-)&=0 \Rightarrow A^\P = B^\P\\
        \p_x\vp^\P(0^+)-\p_x\vp^\I(0^-)&=0 : \text{ toujours satisfait}
    \end{align}
    On obtient donc les solutions suivantes:
    \begin{align}
        \vp^\P(x) &= 2A^\P\cos(kx),\\
        \vp^\I(x) &= 2iA^\I\sin(kx).
    \end{align}
    À ce stade, nous avons obtenu exactement les mêmes expressions qu'à l'étape 2 de la méthode A. Il ne reste plus qu'à imposer les conditions aux bords et la normalisation, exactement comme dans la méthode A, et tous les résultats seront les mêmes.
    \item On impose les conditions aux bords: voir étape 3 de la méthode A.
    \item Nous imposons la normalization : voir étape 4 de la méthode A.
\end{enumerate}
Les solutions finales sont exactement les mêmes que celles obtenues par la méthode A: \eqref{eq:solf1} et \eqref{eq:solf2}.

\subsection{Comparaison des méthodes}

    Dans les deux cas, nous obtenons les solutions
    \begin{align}
        \vp_n^\P(x) &= \sqrt{\frac{2}{L}}\cos(k^\P_nx),\quad k^\P_n=\frac{\pi}{L}(2n+1),\quad (n\in\N),\label{eq:solf1}\\
        \vp_n^\I(x) &= \sqrt{\frac{2}{L}}\sin(k^\I_nx), \quad k^\I_n = \frac{\pi}{L}2n, \qquad (n\in\N_0).\label{eq:solf2}
    \end{align}
    On peut éventuellement regrouper ces deux familles de solutions en une seule: si l'on définit $k_n\equiv\frac{\pi}{L}n$, alors
    \begin{equation}
        k_n=
        \begin{cases}
            k^\P_n ,\quad \text{si $n$ est impair},\\
            k^\I_n ,\quad \text{si $n$ est pair}.
        \end{cases}
    \end{equation}
    On peut alors définir
    \begin{equation}
        \vp_n(x) \equiv
        \begin{cases}
            \sqrt{\frac{2}{L}}\cos(k_nx),\quad \text{si $n$ est impair},\\
            \sqrt{\frac{2}{L}}\sin(k_nx),\quad \text{si $n$ est pair},
        \end{cases}
    \end{equation}
    auquel cas
    \begin{equation}
        \vp_n(x)=
        \begin{cases}
            \vp_n^\P(x),\quad \text{si $n$ est impair},\\
            \vp_n^\I(x),\quad \text{si $n$ est pair}.
        \end{cases}
    \end{equation}
    Autrement dit, on retrouve exactement \eqref{eq:solfin}.

    Nous pouvons conclure que les trois méthodes sont exactement équivalentes. Cependant, cela vient du fait que, dans la méthode B, nous avons pensé à imposé que que la solution soit $C^1$ en $x=0$. C'est ce critère qui nous permet de retrouver les mêmes expression que dans la méthode A. Cette étape n'était pas nécessaire lors des deux autres approches car, par construction\footnote{On résout directement dans le puit entier.}, les solutions sont directement $C^1$ partout.

\section{Ajout d'une barrière infiniment mince}

    Si l'on ajoute une barrière de potentielle infiniment mince au milieu du puit de potentiel, la fonction d'onde n'est plus $C^1$ en $x=0$. Elle est continue, mais sa dérivée est discontinue et la condition de raccord dépend de la barrière de potentielle. Comme la fonction d'onde n'est plus $C^1$, on ne peut pas utiliser la méthode A, ni la méthode ou l'ont ne tient pas compte de la symétrie. En effet, ces méthodes ne peuvent que donner lieu que à des soltuion $C^1$ partout, par construction.

%\textcolor{blue}{(work in progress)}






\end{document}