\documentclass[11pt,a4paper,oneside]{article}

\usepackage{import}

\import{Packages/}{custom_packages.tex}
\import{Packages/}{custom_macros.tex}

\title{L'oscillateur harmonique : Rappel}
\author{Louan Mol}

\begin{document}

\begin{center}
    {\huge \textbf{L'oscillateur harmonique : Rappel}}
\end{center}

\section{Oscillateur harmonique quantique}

\paragraph*{Hamiltonien.} L'Hamiltonien d'une particule de masse $m$ dans un potentiel harmonique de pulsation $\omega$ est
\begin{equation*}
    \hH = \frac{\hP^2}{2m}+\frac{1}{2}m\omega^2\hX^2.
\end{equation*}
Ce dernier peut être généralisé à plusieurs dimensions. Pour rappel, les opérateurs de position et d'impulsion satisfont $[\hX,\hP]=i\hbar\mathbbm{1}$.

\paragraph*{Opérateur d'échelle.} On définit l'\emph{opérateur d'annihilation}
\begin{eqnarray}
    \ha = \sqrt{\frac{m\omega}{2\hbar}}\hX+i\frac{\hP}{\sqrt{2m\hbar\omega}}.
\end{eqnarray}
On peut montrer que
\begin{equation*}
    \boxed{[\ha,\ha^\dagger]=\mathbbm{1}}.
\end{equation*}
L'hermicien conjugué $\ha^\dagger$ est appellé \emph{opérateur de création}. L'Hamiltonien peut être réécrit comme
\begin{align}
    \hH &= \left(\ha^\dagger\ha+\frac{1}{2}\right)\\
    &= \left(\hN+\frac{1}{2}\right)
\end{align}
avec $\hN\equiv\ha^\dagger\ha$. Étudier le spectre de l'Hamiltonien revient donc à étudier le spectre de $\hN$. Nous trouvons les résultats suivants:
\begin{enumerate}[label=\roman*)]
    \item Le spectre de $\hN$ est $\N$ et non-dégénéré. Notons $\ket{n}$ le vecteur propre associé à la valeur propre $n\in\N$.
    \item L'opérateur de création permet de monter dans les valeurs propres:
    \begin{equation*}
        \boxed{\ha^\dagger\ket{n}=\sqrt{n+1}\ket{n+1}}.
    \end{equation*}
    \item L'opérateur de création permet de descendre dans les valeurs propres:
    \begin{equation*}
        \boxed{\ha\ket{n}=\sqrt{n}\ket{n-1}}
    \end{equation*}
    et le ket $\ket{0}$ est annihilé par $\ha$: $\ha\ket{0}=0$.
\end{enumerate}

Le ket $\ket{0}$ est le vecteur propre qui a la plus petite valeur propre. En partant de celui-ci et en agissant avec l'opérateur de création on peut générer tous les vecteurs propres:
\begin{equation*}
    \boxed{\ket{n} = \frac{(\ha^\dagger)^n}{\sqrt{n!}}\ket{0}}.
\end{equation*}

Ces états sont appellés \emph{états de Fock}. Par construction, ils sont aussi états propres de l'Hamiltonien. L'état $\ket{n}$ a une énergie
\begin{equation*}
    E_n = \hbar\omega\left(n+\frac{1}{2}\right).
\end{equation*}
On dit que l'état $\ket{n}$ décrit $n$ ``quantas'' d'énergie, ou $n$ excitations. Remarquons que l'état du vide possède une énergie non-nulle: $E_0=\hbar\omega/2$.

La fonction d'onde associée à l'état $\ket{n}$ est
\begin{equation*}
    \psi_n(x) = \braket{x}{n} = \frac{1}{\sqrt{2^nn!}}\left(x-\dv{x}\right)^n\psi_0(x)
\end{equation*}
ce qui peut être réécrit comme
\begin{equation*}
    \psi_n(x) = \frac{1}{\sqrt[4]{\pi}}\frac{e^{-\frac{x^2}{2}}}{\sqrt{2^nn!}}H_n(x)
\end{equation*}
où $H_n(x)$ est le $n$-ème polynôme d'Hermite.

\paragraph*{Polynômes d'Hermite.} On définit le $n$-ème \emph{polynôme d'Hermite} comme
\begin{equation*}
    H_n(x)=(-1)^ne^{x^2}\dv[n]{x}e^{-x^2}.
\end{equation*}
Grâce à la relation de récurrence
\begin{equation*}
    H_{n+1}(x) = 2xH_n(x)-2nH_{n-1}(x),
\end{equation*} 
il suffit de connaitre explicitement les deux premiers polynômes pour générer les polynômes suivants. Voici les quelques premiers:
\begin{align*}
    H_0(x) &= 1 \\
    H_1(x) &= 2x \\
    H_2(x) &= x^2-1 \\
    H_3(x) &= x^3-3x \\
    H_4(x) &= x^4-6x^2+3,\\
    &\vdots
\end{align*}
Ces polynômes forment une base orthogonale de l'espace de Hilbert $L^2(\R)$. Chaque polynôme est soit paire soit impair et sa parité est la même que celle de $n$.

\section{États cohérents}

\paragraph*{Définition.} Un \emph{état cohérent} $\ket{\alpha}$ est un état propre de l'opérateur d'annihilation:
\begin{equation*}
    \ha\ket{\alpha}=\alpha\ket{\alpha}.
\end{equation*}
Notons que ceci est équivalent à être un vecteur propre de l'opérateur de création avec la valeur propre $\alpha^*$: $\ha^\dagger\ket{\alpha}=\alpha^*\ket{\alpha}$. $\ket{\alpha}$ est carctérisé par sa valeur propre $\alpha\in\C$. On par donc de l'\emph{amplitude} $\abs{\alpha}$ et de la \emph{phase} $\arg{\alpha}$ d'un état cohérent.

Sur la base de Fock, $\ket{\alpha}$ s'écrit comme
\begin{equation*}
    \ket{\alpha} = e^{-\frac{\abs{\alpha}^2}{2}}\sum_{n\in\N}\frac{\alpha^n}{\sqrt{n!}}\ket{n}.
\end{equation*}

\paragraph*{Propriétés physiques.} Les états cohérents ont des propriétés physiques particulières:
\begin{enumerate}[label=\roman*)]
    \item Ils sont inchangés sous l'annihilation ou l'excitation de la particule.
    \item Ils saturent la relation d'incertitude d'Heisenberg.
    \item Leur évolution temporelle est concentrée le long de la trajectoire classique et il n'y a pas de dispersion, contrairement aux états propres d'énergie.
\end{enumerate}











\end{document}