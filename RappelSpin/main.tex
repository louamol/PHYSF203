\documentclass[11pt,a4paper,oneside]{article}

\usepackage{import}

\import{Packages/}{custom_packages.tex}
\import{Packages/}{custom_macros.tex}

\title{Formalisme de Dirac: Rappel}
\author{Louan Mol}

\begin{document}

\begin{center}
    {\huge \textbf{Le spin : Rappel}}
\end{center}

\section{Matrices de Pauli}

\paragraph*{Définition.} Les \emph{matrices de Pauli} sont les trois matrices $2\times2$ à coefficients complexes suivantes:
\begin{equation}
    \sigma_x = 
    \begin{bmatrix}
        0 & 1 \\
        1 & 0
    \end{bmatrix},\qquad
    \sigma_x = 
    \begin{bmatrix}
        0 & -i \\
        i & 0
    \end{bmatrix},\qquad
    \sigma_x = 
    \begin{bmatrix}
        1 & 0 \\
        0 & -1
    \end{bmatrix}.
\end{equation}
En particulier, ce sont toutes les trois des matrices hermitiennes, de trace nulle et de déterminant $-1$.

\paragraph*{Multiplication.} Elles satisfont la propriété importante suivante:
\begin{equation}
    \boxed{\sigma_a\sigma_b=\delta_{ab}\mathbbm{1}+i\varepsilon_{abc}\sigma_c}.
\end{equation}
En particulier, le carré d'une matrice de Pauli est l'identité et le produit de deux matrices de Pauli différentes est à nouveau une matrice de Pauli. À partir de cette relation, on peut montrer que 
\begin{equation}
    \boxed{[\sigma_a,\sigma_b]=2i\varepsilon_{abc}\sigma_c}
\end{equation}
et
\begin{equation}
    \boxed{\{\sigma_a,\sigma_b\}=2\delta_{ab}\mathbbm{1}}.
\end{equation}

\paragraph*{Structure propre.} Notons $\{\ket{e_1},\ket{e_2}\}$ la base dans laquelle les matrices de Pauli sont expirmées.
\begin{table}[H]
    \centering
    $
    \begin{array}{|c|c|c|}
        \hline
        \textbf{Matrice} & \textbf{Vecteurs propres} & \textbf{Valeurs propres} \\ \hline
        \multirow{2}{*}{$\sigma_x$} & \frac{\ket{e_1}+\ket{e_2}}{\sqrt{2}} & +1 \\
        & \frac{\ket{e_1}-\ket{e_2}}{\sqrt{2}} & -1 \\ \hline
        \multirow{2}{*}{$\sigma_y$} & \frac{\ket{e_1}+i\ket{e_2}}{\sqrt{2}} & +1 \\
        & \frac{\ket{e_1}-i\ket{e_2}}{\sqrt{2}} & -1 \\ \hline
        \multirow{2}{*}{$\sigma_z$} & \ket{e_1} & +1 \\
        & \ket{e_2} & -1 \\ \hline
    \end{array}
    $
\end{table}

\paragraph*{Relation avec les matrices hermitiennes.} Les matrices $\sigma_a$ étant hermitiennes et de traces nulles, toute combinaison linéaire réelle
\begin{equation}
    \vec{n}\cdot\vec{\sigma}\equiv n_x\sigma_x+n_y\sigma_y+n_z\sigma_z = 
    \begin{bmatrix}
        n_z & n_x-in_y\\
        n_x+in_y & -n_z
    \end{bmatrix},\qquad \vec{n}\in\R^3
\end{equation}
est aussi hermitienne et de trace nulle. Inversément, on peut montrer que tout matrice $2\times2$ hermitienne et de trace nulle\footnote{L'ensemble de ces matrices est noté $\mathfrak{su}(2)$.} peut s'écrire comme une combinaison linaire réelle des matrices de Pauli.\\

Plus généralement, toute matrice $2\times 2$ heritienne peut s'écrire comme une combinaison linéaire réelle des matrices de Pauli et de l'identité.

\paragraph*{Relation avec les matrices unitaires.} Nous avons vu précédemment que si $A$ est une matrice hermitienne, alors $U=e^{iA}$ est une matrice unitaire. On conclu donc que toute matrice de la forme $e^{i\vec{n}\cdot\vec{\sigma}}$ est unitaire et de déterminant $1$. Qu'en est-il de la question inverse: est-ce que toute matrice $2\times2$ unitaire de déterminant $1$\footnote{L'ensemble des ces matrices est noté $\text{SU(2)}$.} peut s'écrire sous cette forme ? La réponse est oui. Comprendre pourquoi demande cependant plus de formalisme (voir cours sur les groupes de Lie et algèbres de Lie en BA3).

\section{Moment angulaire et spin}

\paragraph*{Moment angulaire.} On définit les \emph{moments angulaires} dans chaque direction d'espace comme trois matrices $J_x,J_y$ et $J_z$ qui satisfont les relations de commutation suivantes:
\begin{equation}
    \boxed{[J_i,J_j] = i\varepsilon_{ijk}J_k}.
\end{equation}
L'action d'une rotation d'angle $\theta$ autour de l'axe $\vec{n}$ est alors représentée par la matrice unitaire
\begin{equation}
    \boxed{\hU(\theta,\vec{n})\equiv e^{i\theta\vec{n}\cdot\vec{J}}}
\end{equation}
et un état se transforme sous rotation de la manière suivante:
\begin{equation}
    \ket{\psi}\mapsto \hU(\theta,\vec{n})\ket{\psi}.
\end{equation}
Il existe une beauoucp de possiblités de matrices $J_i$ qui satisont ces relations. Ce choix (le choix de representation) dépend du spin de l'objet considéré.

\paragraph*{Spin.} Le \emph{spin} est une propriété interne d'un objet (tout comme sa masse, sa charge, etc) caractérisé par un nombre entier ou demi-entier: $s\in \N/2$. \\

L'état physique d'une particule avec spin est la donnée de son état ``interne'' (position, impulsion, etc) et de son état de spin. L'espace de Hilbert est donc ``scindé'' en deux parties: la partie interne et la partie spin. Par exemple, pour une particule de spin $1/2$ à 1 dimension, l'espace de Hilbert est $\H=L^2(\R)\otimes\C^2$. Plus de détails seront donnés dans le cours de BA3. Pour l'instant nous considérerons les deux séparément: soit nous étudions l'état interne de la particule (comme nous l'avons fait jusqu'à présent), soit son état de spin (TP 5).


\paragraph*{Spin $\boldsymbol{1/2}$.} Pour une particule de spin $1/2$, les matrices de moment angulaire sont
\begin{equation}
    S_i = \frac{\hbar}{2}\sigma_i.
\end{equation}
On peut montrer que l'opérateur qui implémente les rotations est alors
\begin{equation}
    U(\theta,\vec{n}) = e^{i\frac{\hbar}{2}\theta \vec{n}\cdot\vec{\sigma}} = \cos\frac{\theta}{2}\mathbbm{1}+i\sin\frac{\theta}{2}\vec{n}\cdot\vec{\sigma}.
\end{equation}

\paragraph*{Moment magnétique.} En mécanique classique, si un objet possède un moment magnétique $\vec{\mu}$, la présence d'un champ magnétique $\vec{B}$ induit un moment de force
\begin{equation}
    \vec{\tau} = \vec{\mu}\times\vec{B}
\end{equation}
sur l'objet. Le moment angulaire $L$ et le moment magnétique $\mu$ sont reliés par le \emph{rapport gyromagnétique}
\begin{equation}
    \gamma\equiv\frac{\mu}{L}.
\end{equation}
Par exemple, considérons le cas d'une particule de charge $q$ et de masse $m$ en MCU de rayon $r$ à vitesse $v$. Il y a à la fois un moment angulaire $L=mrv$ venant du fait que la particule tourne et qu'elle est massive et un moment magnétique $\mu=IA$ venant du fait que la particule tourne et qu'elle est chargée. On trouve rapidement que
\begin{equation}
    \gamma = \frac{q}{2m}.
\end{equation}








\end{document}